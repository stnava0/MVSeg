\documentclass[11pt,english]{article}

% Set page margins correctly
\usepackage{geometry}
\usepackage{url}
\geometry{letterpaper,top=1.0in,left=1.0in,bottom=1.0in,top=1.0in,headsep=6pt,footskip=18pt}

\usepackage{lscape}
\usepackage[square,comma,numbers,sort&compress]{natbib}

% Use fancy header style
\usepackage{fancyhdr}
\pagestyle{empty}
\renewcommand{\headrulewidth}{0.75pt}
\renewcommand{\footrulewidth}{0.75pt}
\usepackage{setspace}
\usepackage{listings}
\usepackage{float}
\floatstyle{plain}
\newfloat{command}{thp}{lop}
\floatname{command}{Command}
\doublespacing
\usepackage{boxedminipage}
\usepackage{graphicx}
\usepackage{amsmath,amsfonts}
\usepackage{babel,verbatim}
\usepackage{enumerate}
\usepackage{longtable}
\usepackage{multirow}
\usepackage{sectsty}
\usepackage[compact]{titlesec}
\usepackage[usenames]{color}
\usepackage{ulem}
\usepackage{multirow,booktabs,ctable,array}
\graphicspath{{./Figures/}
                          }

%\DeclareMathOperator*{\argmax}{arg\,max}
\newcommand{\argmax}{\operatornamewithlimits{argmax}}
\newcommand{\argmin}{\operatornamewithlimits{argmin}}

\long\def\symbolfootnote[#1]#2{\begingroup%
\def\thefootnote{\fnsymbol{footnote}}\footnote[#1]{#2}\endgroup}

    \usepackage{color}

    \definecolor{listcomment}{rgb}{0.0,0.5,0.0}
    \definecolor{listkeyword}{rgb}{0.0,0.0,0.5}
    \definecolor{listnumbers}{gray}{0.65}
    \definecolor{listlightgray}{gray}{0.955}
    \definecolor{listwhite}{gray}{1.0}

\begin{document}
\normalem

\vspace*{5cm}

\begin{center}
{\Large \bf Multivariate EM Segmentation Project} \\
\vspace*{0.5cm}
{\normalsize Ben Kandel, Pengfei Zheng and Brian B. Avants$^1$} \\
\begin{singlespace} 
{\scriptsize  $^1$ Penn Image Computing and Science Laboratory, University of Pennsylvania, Philadelphia, Pennsylvania,  USA.}
\end{singlespace}
\end{center}

\vfill

\begin{singlespace} 
\scriptsize
\flushleft
%\line(1, 0){250} \\
{\bf MVSeg}\\
Corresponding author: \\
Brian B. Avants\\
3600 Market Street, Suite 370\\
Philadelphia, PA  19104\\
avants@picsl.upenn.edu\\
\end{singlespace} 
\clearpage
\begin{abstract} 
Neuroanatomical coordinate systems are essential for the
interpretation of structural and functional imaging studies.
However, manual delineation of the neuroanatomical complex is time
consuming and prone to random performance variability.  This
work describes an open source, image-based approach to white matter parcellation which uses training data to propagate
structural labelings to individual images.  The Bayesian formulation of the segmentation problem is 
solved using the Expectation Maximization (EM) algorithm with a
variety of different multivariate distance metrics.  We evaluate our
ability to segment DTI data based on comparison with
registration-based approaches and biological validity of results. 
\end{abstract}

\section{Introduction}
The Expectation-Maximization (EM) framework \citep{Dempster1977} is a
powerful optimization method for parameter estimation.

\section{Data}

\section{Methods}

\section{Results}

\section{Discussion}

\paragraph{Information Sharing Statement}
{\color{blue}{Atropos software is available in ANTs
    \url{http://www.picsl.upenn.edu/ANTs} 
which depends on ITK \url{http://www.itk.org/Wiki/ITK/Git/Download}.
The data used in this work is available in the ANTs software
repository, BrainWeb 
\url{http://mouldy.bic.mni.mcgill.ca/BrainWeb/} and at \url{www.brain-development.org}.
We employed itk-SNAP for visualization \url{www.itksnap.org}.}}

\paragraph{Acknowledgments}
{This work was supported in part by NIH (AG17586, AG15116, NS44266, and
NS53488).}

\newpage

\bibliographystyle{plain}
\bibliography{MVSeg} 
\end{document}
