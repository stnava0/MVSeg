We evaluate the $N$-class Atropos GMM-MRF model, which includes 
strong use of template-based spatial priors in order to initialize 
and guide the method into a consistent local minimum.
Atropos~encodes a family of segmentation techniques that may be
instantiated for different applications but here we evaluate only two of
the many possibilites. First, we perform an evaluation on the BrainWeb
dataset using both the standard T1 image with multiple bias and noise
levels and also the BrainWeb20 data, which varies the underlying
anatomy.  Second, we evaluate the use of Atropos in improving whole
brain parcellation and excercise its ability to efficiently solve {\em
  many class} expectation maximization problem.  We choose this
evaluation problem in part to illustrate the flexibility of Atropos
and also the benefits of the novel, efficient implementation that
allows many-class problems to be solved with low memory usage ($<$2GB
for a 68 class model on 1mm$^3$ brain data).  


\subsection{BrainWeb Evaluation}
\label{sec:bweb}
 BET2 \citep{Battaglini2008}
against openly available silver-standard datasets, the BrainWeb 20
(BW20) \citep{Aubert-Broche2006a} 


BW20 results show that by relying upon accurate, high-resolution
diffeomorphic image registration \citep{Avants2010}, Atropos~is capable
of reliably segmenting the BW20 whole head data into the brain
composed of separate tissue classes for muscle, skin, skull and bone
marrow as well as the standard three tissues, cerebrospinal fluid and
gray and white matter.  Atropos's three-tissue performance compares
favorably with the standard EM-MRF algorithm, FAST (~Atropos~gray matter
mean dice=0.9442, FAST=0.903,~Atropos~white matter mean dice=0.9475 and
FAST=0.949) \citep{Klauschen2009}.  The brain extraction component of
the solution compares favorably with BET2 (~Atropos~achieved mean $\pm$
s.d. of 0.9642 $\pm$ stdev: 0.0098 dice coefficient and BET2 achieved
0.9682 $\pm$ 0.0068 with an insignificant difference in performance
$p=0.1943$).  

\subsection{The Hammers Dataset Evaluation}
\label{sec:hammers}
We also provide, at the above location, multi-template
labeling results derived from the Hammers dataset by applying the
{\verb ants_multitemplate_labeling.sh } script to the 19 datasets at
\url{http://www.brain-development.org/} \citep{Hammers2003,Heckemann2006}.
Results are competitive with both \citep{Heckemann2006,Heckemann2010} 
though the latter appears to use a different label set.  The closest
comparison may be made with \citep{Heckemann2006}, which uses almost the
same label set, though with 30 datasets in total.  Our results only
incorporated the 19 currently available online. 

\begin{comment}
{
\subsection{Brain Web}

\subsection{The LPBA40 Dataset}  
\label{sec:lpba}
The LONI Probabilistic Brain Atlas data, LPBA40, \citep{Shattuck2008} was collected at the North Shore Long
Island Jewish Health System imaging center and is maintained at UCLA.
LPBA40 contains 40 images (20 male $+$ 20 female) from normal, healthy
ethnically diverse volunteers with average age of 29.2 $\pm$ 6.3
years.  Each subject underwent 3D SPGR MRI on a 1.5T GE system
resulting in $0.86 \times 0.86 \times 1.5 mm^3$ images.  Each MRI in
the LPBA40 dataset was manually labeled with 56 independent structures
at the UCLA Laboratory of Neuro Imaging (LONI).  The test-retest
reliability of the labeling, across raters, was reported as a minimum
Jaccard ratio of $0.697$ in the supramarginal gyrus to a maximum of
$0.966$ in the gyrus rectus.  A single labeling of each image is made
available to the public and used, here, as silver-standard data 
for both training and testing in our cross-validation scheme. 

\subsection{The NIREP Dataset}  
The non-rigid image registration evaluation project (NIREP
http://www.nirep.org/) is a resource of 16 high quality labeled brain
images at 1$mm^3$.  Each brain was labeled with 32 cortical regions
(16 on each hemisphere) and an additional class of other gray matter
tissue.  Regions vary in size from large (inferior temporal region) to
small (temporal pole, insula gyrus, frontal pole).  The main drawback
is a lack of inter-rater reliability numbers -- in particular because
visual inspection and comparison of labelings reveals a degree of
inconsistency in labeling of particular regions across subjects.
Nevertheless, the NIREP dataset is perhaps the highest quality
evaluation dataset currently available for the cortex. 

}
\end{comment}
